\documentclass[pdf]{beamer}

%% Config
\usepackage{fontspec}
\usepackage{polyglossia}
\setmainlanguage{ukrainian}
\setotherlanguage{english}
\newfontfamily\cyrillicfont[Script=Cyrillic]{CMU Serif}
\newfontfamily\cyrillicfontsf[Script=Cyrillic]{CMU Sans Serif}
\newfontfamily\cyrillicfonttt[Script=Cyrillic]{CMU Typewriter Text}

\bibliographystyle{alpha}  
 
% Useful packages
\usepackage{amssymb, amsmath, amsfonts, enumerate, float, indentfirst, graphicx, color}
\usepackage[final]{listings}

\usepackage{tikz}

% Listings
\lstdefinestyle{mystyle}{
  language=python,
  basicstyle=\ttfamily\small,
  showstringspaces=false,
  belowskip=1em,
  aboveskip=1em
%  frame=single,
}
\lstset{escapechar=@,style=mystyle}

%% Title
\title{Перестановки слотів у групованих шифротекстах гомоморфних схем шифрування}
\date[2015]{16 травня 2016}
\author{Богдан Кулинич}

%% Theme
\usetheme{boxes}
\usecolortheme{seahorse}

\setbeamercovered{transparent=20}

\begin{document}

\begin{frame}
\titlepage
\end{frame}


\begin{frame}{}

\begin{itemize}
	
	\item Формалізований опис симетричної схеми гомоморфного шифрування з \cite{YKPB13}
	\item Коректні обмеження параметрів схеми
	\item Реалізація схеми на С++
	
\end{itemize}

\end{frame}


\begin{frame}{Асиметричне шифрування}

Cхема асиметричного шифрування \( \mathcal{E} \):

\begin{itemize}
	
	\item \( \mathsf{KeyGen}_\mathcal{E}( 1^\lambda ) \rightarrow ( \mathbf{pk}, \mathbf{sk} ) \)
	\item \( \mathsf{Encrypt}_\mathcal{E}( \mathbf{pk}, m \in \{0, 1\} ) \rightarrow c \in \mathcal{C}. \)
	\item \( \mathsf{Decrypt}_\mathcal{E}( \mathbf{sk}, c \in \mathcal{C} ) \rightarrow m \in \{ 0, 1 \} \)
	
\end{itemize}

\end{frame}


\begin{frame}{Гомоморфне шифрування}

Cхема асиметричного \textcolor{blue}{гомоморфного} шифрування \( \mathcal{E} \):

\begin{itemize}
	
	\item \( \mathsf{KeyGen}_\mathcal{E}( 1^\lambda ) \rightarrow ( \mathbf{pk}, \mathbf{sk} ) \)
	\item \( \mathsf{Encrypt}_\mathcal{E}( \mathbf{pk}, m \in \{0, 1\} ) \rightarrow c \in \mathcal{C}. \)
	\item \( \mathsf{Decrypt}_\mathcal{E}( \mathbf{sk}, c \in \mathcal{C} ) \rightarrow m \in \{ 0, 1 \} \)
	\textcolor{blue}{
	\item \( \mathsf{Add}_\mathcal{E}( \mathbf{pk} , c_1 \in \mathcal{C}, ~ c_2 \in \mathcal{C} ) \)
	\item \( \mathsf{Mult}_\mathcal{E}( \mathbf{pk} , c_1 \in \mathcal{C}, ~ c_2 \in \mathcal{C} ) \)}
	
\end{itemize}
\end{frame}


\begin{frame}{Відомі схеми гомоморфного шифрування}

\begin{itemize}
\item Над цілими числами: DGHV \cite{DGHV10}
\item \textbf{Над кільцями многочленів: BGV \cite{BGV12}, NTRU \cite{LTV12}}
\item Над матрицями: GSW \cite{GSW13}
\end{itemize}

\end{frame}


\begin{frame}{Групування (batching) шифротекстів}

\begin{enumerate}
\item Китайська теорема про лишки
\end{enumerate}

\end{frame}


\begin{frame}{Висновки}

\begin{itemize}
	\item Виконання змішаних операцій дозволяє не шифрувати \emph{всі} входи функції, що виконується гомоморфно
\end{itemize}
\end{frame}


\begin{frame}{Посилання}
\bibliography{latex-common/bibliography/crypto.bib}{}
\bibliographystyle{plain}
\end{frame}

\end{document}
